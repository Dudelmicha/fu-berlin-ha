\documentclass[a4paper,10pt]{report}

\topmargin -4cm
%\topskip0cm
%\footskip0cm
%\headsep0cm
\parindent0cm
\oddsidemargin -1cm
\evensidemargin -1cm
\headheight 3cm
\textheight 25cm
\textwidth 18cm

\author{Daniel W\"aber (4049590)}
\title{\"Ubung}

\usepackage{ucs}
\usepackage[utf8x]{inputenc}
\usepackage{german}
\usepackage{color}

\pagestyle{empty}
\usepackage{makeidx}
\usepackage{amsmath}
\usepackage{amsfonts}
\usepackage{amssymb,euscript}
\usepackage{dsfont}
\usepackage{listings}
\newfont{\Fr}{eufm10}
\newfont{\Sc}{eusm10}
\newfont{\Bb}{msbm10}
\newcommand{\limin}{\lim_{n\rightarrow\infty}}
\newcommand{\limix}{\lim_{x\rightarrow\infty}}
\newcommand{\limun}{\lim_{n\rightarrow -\infty}}
\newcommand{\limux}{\lim_{n\rightarrow -\infty}}
\newcommand{\limx}{\lim_{x\rightarrow x_0}}
\newcommand{\limh}{\lim_{h\rightarrow 0}}
\newcommand{\defi}{\paragraph{Definition:}}
\newcommand{\bew}{\paragraph{Beweis:}}
\newcommand{\satz}{\paragraph{Satz:}}
\newcommand{\bsp}{\paragraph{Beispiel:}}
\newcommand{\lemma}{\paragraph{Lemma:}}
\newcommand{\N}{\mathds{N}}
\newcommand{\Z}{\mathds{Z}}
\newcommand{\Q}{\mathds{Q}}
\newcommand{\R}{\mathds{R}}
\newcommand{\C}{\mathds{C}}
\newcommand{\K}{\mathds{K}}
\newcommand{\A}{\mathds{A}}
\newcommand{\qed}{$\hfill\blacksquare$}
\newcommand{\arsinh}{\operatorname{arsinh} }
\newcommand{\arcosh}{\operatorname{arcosh} }
\newcommand{\wP}{\mathcal{P} }
\newcommand{\gdw}{$\Leftrightarrow$}
\newcommand{\tf}{$\Rightarrow$}
\newcommand{\mgdw}{\Leftrightarrow}
\newcommand{\mtf}{\Rightarrow}
\newcommand{\Bild}{\text{Bild}}
\newcommand{\Kern}{\text{kern}}
\newcommand{\rg}{\text{rg}}
\newcommand{\deff}{\text{deff}}

\newcommand{\alphato}{\underset{\alpha}\to}
\newcommand{\betato}{\underset{\beta}\to}
\newcommand{\etato}{\underset{\eta}\to}
\newcommand{\ito}{\underset{i}\to}
\newcommand{\sto}{\underset{s}\to}
\newcommand{\kto}{\underset{k}\to}
\newcommand{\xto}{\underset{x}\to}

\usepackage{fancyhdr}
\pagestyle{fancy}
\lhead{Daniel Waeber\\Michael Kmoch}
\chead{"Ubungsblatt \nr\\\today}
\rhead{HA\\Tutor: Claudia Dieckmann}



\newcommand{\nr}{5}

\begin{document}
\section*{Aufgabe 1}
\begin{enumerate}[(a)]
\item AVL-Baum

\begin{itemize}
\item 32: 
\begin{dot2tex}[autosize]
graph G {
    32
}
\end{dot2tex}

\item 27: 
\begin{dot2tex}[autosize]
graph G {
    32 -- 27
    0[style=invis]
    32 -- 0[style=invis]
}
\end{dot2tex}

\item 13: AVL fuer Wurzel verletzt, einfache Rotation noetig
\begin{dot2tex}[autosize]
graph G {
    a32[label=32,shape=octagon]
    a27[label=27]
    a13[label=13]
    a32 -- a27 -- a13
    a0[style=invis]
    a32 -- a0[style=invis]
    a27 -- a0[style=invis]
    27 -- 13
    27 -- 32
}
\end{dot2tex}

\item 41:
\begin{dot2tex}[autosize]
graph G {
    0[style=invis]
    27 -- 13
    27 -- 32
    32 -- 0[style=invis]
    32 -- 41
}
\end{dot2tex}

\item 55: Rotation in Teilbaum noetig
\begin{dot2tex}[autosize]
graph G {
    0[style=invis]
    27 -- 13
    27 -- 32
    32[shape=octagon]
    32 -- 0[style=invis]
    32 -- 41
    41 -- 0[style=invis]
    41 -- 55

    a27[label=27]
    a13[label=13]
    a32[label=32]
    a41[label=41]
    a55[label=55]

    a27 -- a13
    a27 -- a41
    a41 -- a32
    a41 -- a55
}
\end{dot2tex}

\item 86: Doppelrotation in Wurzel noetig
\begin{dot2tex}[autosize]
graph G {
    a0[style=invis]
    a27[label=27,shape=octagon]
    a13[label=13]
    a32[label=32]
    a41[label=41]
    a55[label=55]
    a86[label=86]

    a27 -- a13
    a27 -- a41
    a41 -- a32
    a41 -- a55
    a55 -- a0[style=invis]
    a55 -- a86

    0[style=invis]
    41 -- 27
    27 -- 13
    27 -- 32
    41 -- 55
    55 -- 0[style=invis]
    55 -- 86
}
\end{dot2tex}
\item 72: Doppel-Rotation noetig
\begin{dot2tex}[autosize]
graph G {
    41 -- 27
    27 -- 13
    27 -- 32
    41 -- 55
    0[style=invis]
    55[shape=octagon]
    55 -- 0[style=invis]
    55 -- 86
    86 -- 72
    o[style=invis]
    86 -- o[style=invis]

    a0[style=invis]
    a13[label=13]
    a27[label=27]
    a32[label=32]
    a41[label=41]
    a55[label=55]
    a72[label=72]
    a86[label=86]

    a41 -- a27
    a27 -- a13
    a27 -- a32
    a41 -- a72
    a72 -- a55
    a72 -- a86
}
\end{dot2tex}

\item 69:
\begin{dot2tex}[autosize]
graph G {
    a0[style=invis]
    a13[label=13]
    a27[label=27]
    a32[label=32]
    a41[label=41]
    a55[label=55]
    a69[label=69]
    a72[label=72]
    a86[label=86]

    a41 -- a27
    a27 -- a13
    a27 -- a32
    a41 -- a72
    a72 -- a55
    a72 -- a86
    a55 -- a0[style=invis]
    a55 -- a69
}
\end{dot2tex}

\item Loeschen 72: Der rechte teilbaum ist hoerer, desshalb wird der Vorgaenger von 72, also 69, gewaehlt, um den knoten zu ersetzen
\begin{dot2tex}[autosize]
graph G {
    a0[style=invis]
    a13[label=13]
    a27[label=27]
    a32[label=32]
    a41[label=41]
    a55[label=55]
    a69[label=69]
    a72[label=72]
    a86[label=86]

    a41 -- a27
    a27 -- a13
    a27 -- a32
    a41 -- a72
    a72 -- a55
    a72[label=""]
    a69[shape=rect]
    a72 -- a86
    a55 -- a0[style=invis]
    a55 -- a69

    41 -- 27
    27 -- 13
    27 -- 32
    41 -- 69
    69 -- 55
    69 -- 86
}
\end{dot2tex}
\end{itemize}
\end{enumerate}

\section*{Aufgabe 2}
\begin{enumerate}[(a)]
\item Bei dem Uebergang von $n = 2^k - 1$ zu $n = 2^k$ sind $k+1$ Flips noetig,
    da alle $1$ zu $0$ gesetzt werden muessen und anschliessend die $k$. Stelle auf $1$.
    Es kann also zu $O(2^k)$ Flips kosten.

\item Anfangs sind alle Bits auf $0$. Die Invariante ist also erfuellt.

    Wird eine Operation ausgefuert, stehen 2 Euro zur verfuegung.
    Nun kann bei der hintersten Stelle begonnen werden, die noetigen Bits zu kippen:
    \begin{description}
    \item[Fall 1: Bit ist 0]
    
    Das mit einem Euro kann das Bit gesetzt werden, der andere wird eingezahlt in die Kasse des Bits.
    Die Erhohung ist nun abgeschlossen.

    \item[Fall 2: Bit ist 1]
    
    Das Bit hat also ein Guthaben von 1 Euro. Mit diesem kann es auf 0 gesetzt werden.

    So kann zu dem naechsten Bit uebergegangen werden, da das Guthaben von $2$
    Euro noch immer zur Verfuegung steht.
    \end{description}

    Also koennen bei einem Schritt alle noetigen Bits geflipt werden, ohne dass die Invariante verletzt werden.
    Die Kosten fuer die Erhohung betragen also armortisiert 2.

    \item 
    Um die armotisierten Kosten zu Berechnen, koennen auch die Gesamtkosten alle Schritte bis $n$ betrachtet werden
    und diese anschliesten durch die Anzahl der Schritte geteilt werden.

    In dem Fall des Binaerzahlers koennen die Gesamtkosten fuer jedes Bit berechnet werden:
    Das erste Bit muss in jedem schritt erhoet werden, das 2. nur in jedem 2. \ldots,
    das Letzt bit wird nur einmal gesetzt.

    Es ergibt sich also fuer die Gesamtkosten $K$
    \begin{eqnarray}
    K &\leq& n + \frac{n}{2} + \frac{n}{4} \cdots + 1 = \sum_{i=1}^{\log_2 n} 2^i = 2^{\log_2 n + 1} = 2n
    \end{eqnarray}

    Auch hier ergeben sich armortisierte Kosten von 2 Flips pro Schritt.
\end{enumerate}

\section*{Aufgabe 3}

\end{document}
