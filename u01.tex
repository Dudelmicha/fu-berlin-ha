\documentclass[a4paper,10pt]{report}

\topmargin -4cm
%\topskip0cm
%\footskip0cm
%\headsep0cm
\parindent0cm
\oddsidemargin -1cm
\evensidemargin -1cm
\headheight 3cm
\textheight 25cm
\textwidth 18cm

\author{Daniel W\"aber (4049590)}
\title{\"Ubung}

\usepackage{ucs}
\usepackage[utf8x]{inputenc}
\usepackage{german}
\usepackage{color}

\pagestyle{empty}
\usepackage{makeidx}
\usepackage{amsmath}
\usepackage{amsfonts}
\usepackage{amssymb,euscript}
\usepackage{dsfont}
\usepackage{listings}
\newfont{\Fr}{eufm10}
\newfont{\Sc}{eusm10}
\newfont{\Bb}{msbm10}
\newcommand{\limin}{\lim_{n\rightarrow\infty}}
\newcommand{\limix}{\lim_{x\rightarrow\infty}}
\newcommand{\limun}{\lim_{n\rightarrow -\infty}}
\newcommand{\limux}{\lim_{n\rightarrow -\infty}}
\newcommand{\limx}{\lim_{x\rightarrow x_0}}
\newcommand{\limh}{\lim_{h\rightarrow 0}}
\newcommand{\defi}{\paragraph{Definition:}}
\newcommand{\bew}{\paragraph{Beweis:}}
\newcommand{\satz}{\paragraph{Satz:}}
\newcommand{\bsp}{\paragraph{Beispiel:}}
\newcommand{\lemma}{\paragraph{Lemma:}}
\newcommand{\N}{\mathds{N}}
\newcommand{\Z}{\mathds{Z}}
\newcommand{\Q}{\mathds{Q}}
\newcommand{\R}{\mathds{R}}
\newcommand{\C}{\mathds{C}}
\newcommand{\K}{\mathds{K}}
\newcommand{\A}{\mathds{A}}
\newcommand{\qed}{$\hfill\blacksquare$}
\newcommand{\arsinh}{\operatorname{arsinh} }
\newcommand{\arcosh}{\operatorname{arcosh} }
\newcommand{\wP}{\mathcal{P} }
\newcommand{\gdw}{$\Leftrightarrow$}
\newcommand{\tf}{$\Rightarrow$}
\newcommand{\mgdw}{\Leftrightarrow}
\newcommand{\mtf}{\Rightarrow}
\newcommand{\Bild}{\text{Bild}}
\newcommand{\Kern}{\text{kern}}
\newcommand{\rg}{\text{rg}}
\newcommand{\deff}{\text{deff}}

\newcommand{\alphato}{\underset{\alpha}\to}
\newcommand{\betato}{\underset{\beta}\to}
\newcommand{\etato}{\underset{\eta}\to}
\newcommand{\ito}{\underset{i}\to}
\newcommand{\sto}{\underset{s}\to}
\newcommand{\kto}{\underset{k}\to}
\newcommand{\xto}{\underset{x}\to}

\usepackage{fancyhdr}
\pagestyle{fancy}
\lhead{Daniel Waeber\\Michael Kmoch}
\chead{"Ubungsblatt \nr\\\today}
\rhead{HA\\Tutor: Claudia Dieckmann}



\newcommand{\nr}{2}

\begin{document}
\section*{Aufgabe 1}
\begin{eqnarray}
&& R_0 = n\\
&& R_1 = 1\\
&& GLEZ\ R_0 \text{ End} \text{ // if $n==0$ return 1}\\
&& R_1 = 2\\
\text{Loop:}&&\\
    && R_2 = R_0 - 1\\
    && GEZ\ R_2 \text{ End} \text{ // if $n==1$ return $r_1$}\\
    && R_2 = R_0 / 2 \\
    && R_2 = R_2 \cdot 2 \\
    && R_2 = R_0 - R_2 \text{ // $r_2 = n \mod 2$}\\
    && GEZ\ R_2 \text{ Quad} \text{ // if $r_2 \neq 0$ }\\
\text{Mul:}&& \text{ // then }\\
    && R_1 = R_1 \cdot 2\\
    && R_0 = R_0 - 1 \\
    && GOTO \text{ Loop}\\
\text{Quad:}&& \text{ // else }\\
    && R_1 = R_1 \cdot 2\\
    && R_0 = R_0 / 2\\
    && GOTO \text{ Loop}\\
\text{End:}&&\\
&& HALT
\end{eqnarray}

\begin{enumerate}[(a)]
\item \ 

\begin{tabular}[h]{l|lll}
Zeile & Kosten & \# & Gesamt\\\hline
1-4  &  4 & 1 & 4\\
6-11 &  6 & $\lceil \log_2 n \rceil$ & $6 \cdot \lceil \log_2 n \rceil$ \\
13-15/17-19 &  2 & $\lceil \log_2 n \rceil$ & $2 \cdot \lceil \log_2 n \rceil$ \\\hline
&&& $4+8\cdot \lceil \log_2 n \rceil$
\end{tabular}

\item
Kosten bei $i$-tem Durchlauf durch Quad, $j$-tem Durchlauf durch Mul:
Im worst-case wird der Mul-Teil genauso oft wie der Quad-Teil ausgefuert,
im best-case gar nicht. Da die Kosten der Mul- und Quad-Abschnitte die gleiche Laufzeit haben,
ist im worst-case die Laufzeit hoechstens doppelte so hoch wie im best-case.
Es recht also den best-case zu Betrachten um das asymtotische Verhalten zu bestimmen:

\begin{tabular}[h]{l|lll}
Zeile & Kosten & \# & Gesamt\\\hline
1 & 1 + $\log n$ & 1 & 1 + $\log n$\\
2 & 1 + 1 & 1 & 2\\
3 & $1 + \log n$ & 1 & $1 + \log n$\\
4 & 1 + 1 & 1 & 2 \\
6 & $1+1 + \log (n/2^i)$ & $\log n$\\
7 & $1 + \log (n/2^i)$ & $\log n$\\
8 & $1+1+\log (n/2^i)+1$ & $\log n$\\
9 & $1+1+\log (n/2^{i-1})+1 $ & $\log n$\\
10 & $1+1+\log (n/2^i)\cdot 2$ & $\log n$\\
11 & $1 + \log (n/2^i) $ & $\log n$\\
17 & $1+1 + \log (2^i) + 1 $ & $\log n$\\
18 & $1+1 + \log (n/2^i) + 1 $ & $\log n$\\\hline
\end{tabular}

Die Gesamtkosten setzen sich aus $k \cdot \log (n/2^i)$ und $\log 2^i$ zusammen:
\begin{eqnarray}
T(n) &\leq& \sum_{i=1}^{\log n} k \cdot \log (n/2^i) + \log 2^i \\
     &=& \sum_{i=1}^{\log n} k \cdot \log n - ki \cdot \log 2 + i \log 2 \\
     &=& n \log n - (k-1) \cdot \log 2\cdot  \sum_{i=1}^{\log n} i \\
     &=& n \log n - k' \cdot \frac{\log 2 \cdot (\log n - 1)}{n} \\
\end{eqnarray}

Die Laufzeit betraegt also $\Omega(n \log n)$

\end{enumerate}

\section*{Aufgabe 2}
\begin{enumerate}[(a)]
\item 

Zur Berechnung wird der Stack zuerst fuer den Aufruf von $Fib(n-1)$ aufgebaut und hat eine 
Groesse von $O(n)$. Wird dannach durch die Abbruchbedinung ein Ergebniss nach oben gegeben und anschliesend $Fib(n-2)$ 
berechnet, wird dadurch weniger Stack verbracht als durch die vorherige
Berchnung. Also ist die Platzkomplexitaet insgesamt auch $O(n)$.

\item

\begin{description}
\item[IA] $n=1, n=0$\\
 \begin{equation}
    T(1) = const \leq c \cdot T(\frac{1+\sqrt{5}}{2})
 \end{equation}
\item[IS] $n-1 \land n-2 \to n$\\
    \begin{eqnarray}
    T(n) &\leq& T(n-1) + T(n-2) \\
         &=& c \cdot ( (\frac{1+\sqrt 5}{2})^{n-1} + (\frac{1+\sqrt 5}{2})^{n-2}) ) \\
         &=& c \cdot ( (\frac{1+\sqrt 5}{2})^{n-2} \cdot (1 + (\frac{1+\sqrt 5}{2})^{n-2}) ) \\
         &\leq& c \cdot (\frac{1+\sqrt 5}{2})^{n-1}
    \end{eqnarray}
\end{description}


\item Durch dynamisches Programmieren kann eine lineare Laufzeit erreicht werden, 
    da nun nur eine Iteration bis $n$ noetig ist.

\begin{verbatim}
fibs = new Array
fibs[0] = 1
fibs[1] = 1
for i=2 to n do
  fibs[i] = fibs[i-1]+fibs[i-2]
return fibs[n-1]
\end{verbatim}


\item 
Durch potenzieren wie in Aufgabe 1 kann auch Logaritmische Laufzeit errecht werden.
\begin{verbatim}
m = [1 1 ; 1 0 ]
a = copy of m
while x > 1
  if x%2 == 0
    a = a*a
    x /= 2
  else
    a = a*m
    x -= 1
return a[0][0]
\end{verbatim}
\end{enumerate}

\section*{Aufgabe 3}
\begin{enumerate}[(a)]
\item 
\begin{verbatim}
min = -oo
max = +oo
for a in list
  if (a < min)
     min = a
  elif (a > max)
     max = a
return min,max
\end{verbatim}

Da die Liste unsortiert ist, und auch sonst keine Annahmen über die Liste
getroffen werden können, ist es lediglich möglich, das triviale Suchen nach dem
Maximum und das triviale Suchen nach dem Minimum in einem Durchlauf
zusammenführen.

\item
\begin{verbatim}
int max = -oo
int max' = -oo
for a in list
  if (a >= max)
     max' = max
     max = a
return max,max'
\end{verbatim}

Hier gilt selbiges wie für 3a) es gibt keine Annahmen über die Liste. Damit
lässt es sich nur mit einem doppelten trivialen Suchen nach dem Maximum in
einem Durchgang lösen. Und zwar in dem man das erste Maximum benutzt um ein
Maximum zu suchen, und wenn man ein neuen Wert findet der größer als das
vorherige Maximum ist, das alte Maximum in max2 schiebt

\item

Zur Bestimmung einer unteren Schranke beim vergleichsbasierten Suchen, 
kann zu einem Algorithmus der vergleichsbaum betrachtet werden.

An dessen Blaettern steht die Position $i$ des gefundenen elements $a_i = b$.
Da es $n$ verschidene solche Elemente gibt, muss dieser Baum $n$ Blaetter besitzen.
Da der Vergleichsbaum ein Binaerbaum ist, muss dieser mindestens eine Hoehe von $log n$ besitzen,
um $n$ Blaetter zu haben. Somit ist die untere Schranke $\Theta(n)$.

Im Falle einer unsortierten Folge kann die Schranke besser bestimmt werden: 
Man nehme an, es gaebe einen Algorithmus, der zu jeder Eingabe weniger als $n$ Vergleiche brauechte.
Nun fixiert man die Eingabefolge $S$ und die Entscheidungen des Zufalls, falls es sich um einen randomisierten Algorithmuss
handelt. Wenn dieser weniger als $n-1$ Vergleiche benoetigt, exisitert Element $a_s,a_t \in S$, die nicht
betrachtet wurden. Laesst man den Algorithmus nun auf $S'$ mit $a_s' = \max S + 1$ und $a_t' = \max S + 2$ nach 
$a_t$ suchen, kann dieser nicht Entschiden welches der beiden Elemente betrachtet werden soll.

\end{enumerate}

\end{document}
