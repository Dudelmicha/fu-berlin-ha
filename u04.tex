\documentclass[a4paper,10pt]{report}

\topmargin -4cm
%\topskip0cm
%\footskip0cm
%\headsep0cm
\parindent0cm
\oddsidemargin -1cm
\evensidemargin -1cm
\headheight 3cm
\textheight 25cm
\textwidth 18cm

\author{Daniel W\"aber (4049590)}
\title{\"Ubung}

\usepackage{ucs}
\usepackage[utf8x]{inputenc}
\usepackage{german}
\usepackage{color}

\pagestyle{empty}
\usepackage{makeidx}
\usepackage{amsmath}
\usepackage{amsfonts}
\usepackage{amssymb,euscript}
\usepackage{dsfont}
\usepackage{listings}
\newfont{\Fr}{eufm10}
\newfont{\Sc}{eusm10}
\newfont{\Bb}{msbm10}
\newcommand{\limin}{\lim_{n\rightarrow\infty}}
\newcommand{\limix}{\lim_{x\rightarrow\infty}}
\newcommand{\limun}{\lim_{n\rightarrow -\infty}}
\newcommand{\limux}{\lim_{n\rightarrow -\infty}}
\newcommand{\limx}{\lim_{x\rightarrow x_0}}
\newcommand{\limh}{\lim_{h\rightarrow 0}}
\newcommand{\defi}{\paragraph{Definition:}}
\newcommand{\bew}{\paragraph{Beweis:}}
\newcommand{\satz}{\paragraph{Satz:}}
\newcommand{\bsp}{\paragraph{Beispiel:}}
\newcommand{\lemma}{\paragraph{Lemma:}}
\newcommand{\N}{\mathds{N}}
\newcommand{\Z}{\mathds{Z}}
\newcommand{\Q}{\mathds{Q}}
\newcommand{\R}{\mathds{R}}
\newcommand{\C}{\mathds{C}}
\newcommand{\K}{\mathds{K}}
\newcommand{\A}{\mathds{A}}
\newcommand{\qed}{$\hfill\blacksquare$}
\newcommand{\arsinh}{\operatorname{arsinh} }
\newcommand{\arcosh}{\operatorname{arcosh} }
\newcommand{\wP}{\mathcal{P} }
\newcommand{\gdw}{$\Leftrightarrow$}
\newcommand{\tf}{$\Rightarrow$}
\newcommand{\mgdw}{\Leftrightarrow}
\newcommand{\mtf}{\Rightarrow}
\newcommand{\Bild}{\text{Bild}}
\newcommand{\Kern}{\text{kern}}
\newcommand{\rg}{\text{rg}}
\newcommand{\deff}{\text{deff}}

\newcommand{\alphato}{\underset{\alpha}\to}
\newcommand{\betato}{\underset{\beta}\to}
\newcommand{\etato}{\underset{\eta}\to}
\newcommand{\ito}{\underset{i}\to}
\newcommand{\sto}{\underset{s}\to}
\newcommand{\kto}{\underset{k}\to}
\newcommand{\xto}{\underset{x}\to}

\usepackage{fancyhdr}
\pagestyle{fancy}
\lhead{Daniel Waeber\\Michael Kmoch}
\chead{"Ubungsblatt \nr\\\today}
\rhead{HA\\Tutor: Claudia Dieckmann}



\newcommand{\nr}{4}

\begin{document}
\section*{Aufgabe 1}
\begin{lstlisting}
int interpolationssuche(double[] S, int l, int r, double a) {
    if (S[l] <= a && a >= S[r]) { // liegt in intervall
        // interpoliere
        double varianz = (S[r] - S[l]);
        int k = l + (int) (((double) r - l) * (a - S[l]) / varianz);
        // suche weiter je nach ergebniss
        if (a > S[k]) {
            return interpolationssuche(S, k + 1, r, a);
        } else if (a < S[k]) {
            return interpolationssuche(S, l, k - 1, a);
        } else {
            return k;
        }
    } else { // nicht gefunden
        return -1;
    }
}
\end{lstlisting}
\begin{lstlisting}
int quadbin(double[] S, int l, int r, double a){
    if(l>=r)
        return -1;
    //Interpoliere um wahrscheinliche Position zu ermitteln
    int locall = l;
    int localr = r;
    double varianz = (S[r] - S[l]);
    int interpol = l + (int) (((double) r - l) * (a - S[l]) / varianz);

    if(S[interpol]<a){ //das gesuchte a liegt rechts der interpolierten Stelle
        locall=interpol;
        localr=(int) (locall+ Math.sqrt(r-l+1));
        do { // suche nach sqrt(n) grossem intervall, indem a liegt
            locall = localr;
            localr = (int) (locall+ Math.sqrt(r-l+1));
            if(localr>=S.length){
                    localr=S.length-1;
            }
        } while(S[localr]<a);
    }else if(S[interpol]>a){//das gesuchte a liegt links der interpolierten Stelle
        localr=interpol;
        locall=(int) (localr - Math.sqrt(r-l+1));
        do { // suche nach sqrt(n) grossem intervall, indem a liegt
            localr = locall;
            locall = (int) (localr - Math.sqrt(r-l+1));
            if(locall<0){
                locall=0;
            }
        } while(S[locall]>a);	
    } else {
        return interpol;
    }
    return quadbin(S, locall, localr, a);
}
\end{lstlisting}

\paragraph{Output}
\begin{verbatim}
generating random arrays
sorting array
starting serach
noetige Vergleiche 11 interpolationssuche
noetige Vergleiche 6 quad. binsuche
3648295 0.3650047047171787
3648295 0.3650047047171787
generating random arrays
sorting array
starting serach
noetige Vergleiche 8 interpolationssuche
noetige Vergleiche 12 quad. binsuche
3520185 0.3519754393375424
3520185 0.3519754393375424
generating random arrays
sorting array
starting serach
noetige Vergleiche 8 interpolationssuche
noetige Vergleiche 8 quad. binsuche
6085719 0.6084726611547623
6085719 0.6084726611547623
generating random arrays
sorting array
starting serach
noetige Vergleiche 11 interpolationssuche
noetige Vergleiche 9 quad. binsuche
9653000 0.9652713157333066
9653000 0.9652713157333066
generating random arrays
sorting array
starting serach
noetige Vergleiche 8 interpolationssuche
noetige Vergleiche 6 quad. binsuche
8480422 0.8479237443782034
8480422 0.8479237443782034
generating random arrays
sorting array
starting serach
noetige Vergleiche 11 interpolationssuche
noetige Vergleiche 12 quad. binsuche
3839678 0.383870533974631
3839678 0.383870533974631
generating random arrays
sorting array
starting serach
noetige Vergleiche 14 interpolationssuche
noetige Vergleiche 6 quad. binsuche
2765306 0.2766251266738775
2765306 0.2766251266738775
\end{verbatim}

\section*{Aufgabe 2}

\begin{enumerate}[(a)]
\item Es muss erst eine obere Grenze gefunden werden, bei der der Lastwagen zerstoert wird. Dannach kann nach der 
    Tonnenanzahl binaer gesucht werden. Zur Bestimmung der oberne Grenze kann mit 1T beginnent, der wert immer verdoppelt werden.

    Haelt der Lastwagen $n$ Tonnen aus, so benoetigt man $\lceil \log_2 n \rceil$ Versuche, um die obere Grenze zu finden.
    Dannauch muss ein Raum von maximal $\frac{1}{2}n$ binaer durchsuchtwerden. Also benoetigt man insgesamt
    maximal $2 \lceil \log_2 n \rceil - 2$ Versuche.

\item Darf nur ein Anhaenger zerstoert werden, muss die bei 1T beginend, jede beladung getestet werden.
    Es sind allso $n$ Versuche notwendig.

\item Bei zwei Anhaengern kann wie unter (a) die obere Grenze bestimmt werden, anschliesend muss noch der maxiamal
    $\frac{1}{2}n$ grosse Raum linear durchsuchtwerden. Bei $k$ Anhaengern kann dieser noch in $k-1$ Schritten auf $\frac{1}{2^{k-1}}$ 
    eingeschraengt werden, wobei maximal $\log \frac{n}{2}$ schritte noetig sind.

    Es werden also $\lceil \log_2 n \rceil + \max \{ k-1 + \frac{n}{2^{k-1}}, \lfloor \log_2 n \rfloor - 2\}$

\end{enumerate}

\end{document}
